% CS410 Counting, Probability, and Computing - HW3
% Russell Miller

\documentclass{article}
\usepackage{anysize}
%\usepackage{wasysym}
\usepackage{graphicx}

%for boxed - note: cannot use alongside package 'wasysym'
\usepackage{amsmath}

\marginsize{2cm}{2cm}{2cm}{2cm}

\title{CS410 HW1}
\author{Russell Miller}
\date{\today}

\begin{document}

\maketitle

\paragraph{1.8 I choose a number uniformly at random from the range [1,
1,000,000]. Using the inclusion-exclusion principle, determine the
probability that the number chosen is divisible by one or more of 4, 6, and
9.\\}
I wasn't sure exactly how to get these sets of numbers, so I wrote a
script in Python that generated them. I checked each set of numbers to get
intersects, and produced a size of each set. Let each variable $divX$
represent how many numbers are divisible by $X$, with multiple $X$s being 
the intersects, and $mil$ just being the number $1,000,000$. Here is the 
final formula I used to get this probability.
\begin{eqnarray*}
P(div4 \cup div6 \cup div9) & = & P(div4) + P(div6) + P(div9) - P(div4 \cap
  div6) - P(div4 \cap div9) - \\
      &   & P(div6 \cap div9) + P(div4 \cap div6 \cap 
  div9)\\
      & = & \frac{div4}{mil} + \frac{div6}{mil} + 
  \frac{div9}{mil} - \frac{div46}{mil} - \frac{div49}{mil} - 
  \frac{div69}{mil} + \frac{div469}{mil}\\
      & = & \frac{250000}{mil} + \frac{166666}{mil} + 
  \frac{111111}{mil} - \frac{83333}{mil} - \frac{27777}{mil} - 
  \frac{55555}{mil} + \frac{27777}{mil}
\end{eqnarray*}
\begin{center}
$\boxed{= .3\bar{8}9}$
\end{center}
{\footnotesize See attached transcript of Pythons script.}

\paragraph{1.10 I have a fair coin and a two-headed coin. I choose one of
the two coins randomly with equal probability and flip it. Given that the 
flip was heads, what is the probability that I flipped the two-headed 
coin?\\}
Let $A$ be the event of choosing the two-headed coin. Let $B$ be the event
of flipping a heads. We're told that $P(B) = \frac{1}{2}$. Getting a heads 
while flipping the two-headed coin is 1. So $P(B \mid A) = 1$. Since there
are 4 total possible outcomes (Heads, Heads, Heads, Tails), getting Heads 
is $\frac{3}{4}$, which is $P(A)$.
\begin{eqnarray*}
  P(A \mid B) & = & \frac{P(A \cap B)}{P(B)}\\
              & = & \frac{P(B \mid A)P(A)}{P(B)}\\
              & = & \frac{1 \times \frac{3}{4}}{\frac{1}{2}}
\end{eqnarray*}
\begin{center}
$\boxed{= \frac{3}{2}}$
\end{center}

\paragraph{1.15 Suppose that we roll ten standard six-sided dice. What is 
the probability that their sum will be divisible by 6, assuming that the 
rolls are independent? (\emph{Hint:} Use the principle of deferred 
decisions, and consider the situation after rolling all but one of the 
dice.)\\}
At the point when 9 dice have been rolled, the outcome is already 
determined. If the sum were 54 (all 9 rolls were 6), then the only possible 
roll that could result in the total sum being divisible by 6 is a 6. 
Similarly, any other sum that the first 9 are capable of have only one 
valid accompanying 10^{th} roll.
\begin{center}
$\boxed{\mbox{Thus the probability of a sum that is divisible by 6 for 10
dice rolls is at most } \frac{1}{6}.}$
\end{center}

\end{document}