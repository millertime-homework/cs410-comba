\documentclass{article}
\usepackage{amsmath,amssymb,latexsym,amsthm}
\usepackage{fullpage}
\newtheorem{lemma}{Lemma}
\author{Thomas Schreiber \& Russell Miller}
\title{HW 2, Fall 2011}
 
\begin{document}
\maketitle
\begin{center}
\hrule
\end{center}
\vspace*{1ex}

\paragraph{Problem 1:} You work at a coffee shop, and you are bored.
To pass the time, you decide to count the number of ways to make~$r$
cents of change out of pennies (1 cent), nickels (5 cents), dimes (10
cents) and quarters (25 cents).  Give the generating function for this. \\

We begin by finding the generating functions for the individual coins: \\

Pennies count individual cents: ${(1 + x + x^2 + x^3 + ...)}$ \\

Nickels count in increments of 5: ${(1 + x^5 + x^{10} + x^{15} + ...)}$ \\

Dimes count in increments of 10: ${(1 + x^{10} + x^{20} + x^{30} + ...)}$ \\

Quarters count in increments of 25: ${(1 + x^{25} + x^{50} + x^{75} + ...)}$ \\

Next, we find the generating function for our coffee problem by multiplying all of the above generating functions together. \\

\begin{eqnarray*}
  a_r & = & (1 + x + x^2 + x^3 + ...) (1 + x^5 + x^{10} + X^{15} + ...) \\
    & \times & (1 + x^{10} + x^{20} + x^{30} + ...) (1 + x^{25} + x^{50} + x^{75} + ...) \\
\end{eqnarray*}

\begin{center}
  ${\boxed{ r = ( \frac{1}{1 - x} ) ( \frac{1}{1 - x^{5}} ) ( \frac{1}{1 - x^{10}} ) ( \frac{1}{1 - x^{25}} ) } }$
\end{center}

\pagebreak

\paragraph{Problem 2:} \textbf{(a)} Give the generating function for $a_n$, the
number of ways to build a set of~$n$ objects from ten distinct types,
where you may choose to have 0,3 or 8 of each type.  \textbf{(b)}
Determine $a_{25}$. \\

\subparagraph{(a)}{ The generating function for choosing 0, 3 or 8 from a single type is: \\ 
  \begin{center}
    ${ ( 1 + x^3 + x^8 ) }$ \\
  \end{center}

And we have 10 types to choose from making our equation: \\

\begin{center}
  ${  \boxed { ( 1 + x^{3} + x^{8} )^{10} } }$
\end{center}
}

\subparagraph{(b)} { What we want are the number of ways to get 25 objects by selecting 0, 3 or 8 from 10 distinct groups. There is only one way that multiples of 3 and 8 can equal 25 ( ${ ( 3 * 3 ) + ( 8 * 2 ) = 25 }$. So, we need to choose 2 groups to select 8 objects from and 3 groups to select 3 objects from. \\

\begin{center}
  ${ \boxed { a_{25} = {10 \choose 2}  {8 \choose 3}  = 2520 } }$ \\
\end{center}

\paragraph{Problem 3:} Show that $(1-x-x^2-x^3-x^4-x^5-x^6)^{-1}$ is
the generating function for the number of ways a sum of~$r$ can occur
if a die is rolled an arbitrary number ($0,1,2,\ldots$) of times. \\

The generating function for a single die is: \\
\begin{center}
  ${ ( x + x^2 + x^3 + x^4 + x^5 + x^6 ) }$\\
\end{center}

We want the sum of an arbitrary number of dice: \\
\begin{eqnarray*}
 a_r  & = &  \sum_{i=0}^{\infty} ( x + x^2 + x^3 + x^4 + x^5 + x^6 )^i \\
      & = &  \sum_{i=0}^{\infty} y^i,\ where \  y = ( x + x^2 + x^3 + x^4 + x^5 + x^6 ) \\
      & = &  \frac{1}{1 - y} \\
      & = &  \frac{1}{1 - ( x + x^2 + x^3 + x^4 + x^5 + x^6 ) }\\
      & = &  \frac{1}{(1 - x - x^2 - x^3 - x^4 - x^5 - x^6 ) }  \\ \\
      & = &  (1 - x - x^2 - x^3 - x^4 - x^5 - x^6 )^{-1}  \\
      \blacksquare
\end{eqnarray*}


\paragraph{Problem 4:} Use generating functions to solve the
recurrence relation $a_n = 4a_{n-1} + (-2)^n$ for $n \geq 1$, with
$a_0=1$.  Make sure to give a closed-form solution for $a_n$.  \\

Given: 
\begin{eqnarray*}
  a_0 & = & 1 \\
  a_n & = & 4 a_{n - 1} + (-2)^n 
\end{eqnarray*}
I will find a solution to the recurrence relation using the generating function A(x).

\begin{eqnarray}
  A(x ) & = & \sum_{n=0}^{\infty} a_nx^n 
\end{eqnarray}
Now I'm substituting $a_n$ back in.

\begin{eqnarray}
 & = & \sum_{n=1}^{\infty} (4a_{n-1}x^n + (-2)^nx^n) \\
 & = & \sum_{n=1}^{\infty} (4a_{n-1}x^n) + \sum_{n=0}^{\infty} (-2x)^n \\
 & = & 4x \sum_{n=0}^{\infty} (a_{n}x^n) + \sum_{n=0}^{\infty} (-2x)^n
\end{eqnarray}
Let $y = -2x$.

\begin{eqnarray}
& = & 4x \sum_{n=0}^{\infty} (a_{n}x^n) + \sum_{n=0}^{\infty} y^n\\
& = & 4x A(x) + \frac{1}{1-y}
\end{eqnarray}
Now I'm substituting $y$ back in.

\begin{eqnarray}
& = & 4x A(x) + \frac{1}{1+2x} \\
A(x)(1-4x) & = & \frac{1}{1+2x} \\
A(x) & = & \frac{1}{1+2x}\frac{1}{1-4x}
\end{eqnarray}
Now using partial fractions, we can turn this product into a sum.

\begin{eqnarray}
1 & = & \frac{a}{1+2x} + \frac{b}{1-4x}
\end{eqnarray}
Assume $a+b=1$.

\begin{eqnarray}
1 & = & a(1-4x) + b(1+2x) \\
1 & = & a - 4ax + b + 2bx \\
0 & = & -4ax + 2bx \\
2b & = & 4a \\
b & = & 2a \\
a & = & \frac{1}{3} \\
b & = & \frac{2}{3}
\end{eqnarray}
Now we can put $a$ and $b$ back into the $A(x)$ equation.

\begin{eqnarray}
A(x) & = & \left(\frac{1}{3}\right)\frac{1}{1+2x} + \left(\frac{2}{3}\right)\frac{1}{1-4x}
\end{eqnarray}
Now to get the $a_n$ term.

\begin{eqnarray}
a_n & = & \frac{1}{3}(-2)^n + \frac{2}{3}(4)^n
\end{eqnarray}

\begin{center}
$\boxed{a_n = \frac{1}{3}((-2)^n + 2^{2n+1})}$
\end{center}

\end{document}