% CS410 Counting, Probability, and Computing - HW1
% Thomas Schreiber and Russell Miller

\documentclass{article}
\usepackage{anysize}
%\usepackage{wasysym}
\usepackage{graphicx}

%for boxed - note: cannot use alongside package 'wasysym'
\usepackage{amsmath}

\marginsize{2cm}{2cm}{2cm}{2cm}

\title{CS410 HW1}
\author{Thomas Schreiber and Russell Miller}
\date{\today}

\begin{document}

\maketitle

\paragraph{1. }
\textbf{Each of ten people simultaneously requests a distinct movie from a new service that broadcasts 3D movies. These 3D movies are broadcast as a left and a right stream, which are distinct. (So 10 distinct movies, 20 distinct streams in total.) But the service is not very reliable, and the 3D effect only works if you receive a matching left and right stream. Assume that each person receives exactly two of the streams, and each stream goes to exactly one person.}
\begin{enumerate}
\item[a.] \textbf{How many ways are there for the streams to be broadcast if the service provider makes no guarantees about which streams each person receives?}\\
Because we are organizing the 20 possible streams in 20 possible orders, we use the permutation formula to see the answer is: $\boxed{20!}$
\item[b.] \textbf{How many ways if each person receives a matching left and right stream?}\\
This time we simply pick which movie we are watching (10!), then pick whether the left stream comes to the left side of the TV ($2^{10}$). Using the multiplication rule the solution is: $\boxed{(10!)(2^{10}) = 3715891200}$
\item[c.] \textbf{How many ways if each person is guaranteed one left stream and one right stream? (Not necessarily for the same movie.)}\\
This time we have 2 groups of streams, the left ones and the right ones. First we pick one from the left (10!), then one from the right (10!), then pick whether the left stream comes to the left side of the TV ($2^{10}$). The total product would be:
$\boxed{(10!)(10!)(2^{10})}$
\end{enumerate}

\paragraph{2. }
\textbf{If three distinct dice are rolled, what is the probability that the largest rolled value is exactly twice the smallest?}\\
There are 3 cases where this is satisfied.
\begin{itemize}
\item There is at least one 6, at least one 3, and the third cannot be 1 or 2. After placing the 6 and the 3, we need to choose 1 number from the 4 that are available ${4 \choose 1} = 4$. There are multiple orderings of the dice. When two dice have the same number there are 3 possible variations, and when all three dice are different there are 6 possible variations. So, ${3 + 6 + 6 + 3 = 18}$.
\item There is at least one 4, at least one 2, and the third can only be 2, 3, or 4. After placing the 4 and the 2, we need to choose 1 number from the 3 that are available ${3 \choose 1} = 3$. After accounting for variations there are ${3 + 6 + 3 = 12}$ possibilities.
\item There are only 1's and 2's. After placing the required 1 and 2, we need to choose 1 number that is either 1 or 2 ${2 \choose 1} = 2$. When we account for variations on the ordering we get ${3 + 3 = 6}$
\end{itemize}
Because these are disjoint, the number of ways is obtained with a sum, divided by the total number of ways to roll 3 dice to get the probability.\\
$\boxed{\left( \frac{18+12+6}{6^3} \right) = \frac{1}{6}}$

\pagebreak

\paragraph{3. }
\textbf{Give an exact expression for the probability that two (or more) people in a group of 25 have the same birthday?}\\
Including the extra day of a leap year there are 366 possible birthdays. We are calculating the probability of everyone having different birthdays. We started by selecting one person and calculating the probability that no one that has been selected yet has the same birthday as them. For the first person it's $\frac{366}{366} = 1$. For the second person, we calculate the probability that their birthday is different from the first ($\frac{365}{366}$) and so on until the last person, who has a probability of $\frac{341}{366}$ of not sharing a birthday with anyone in the room. The complement of "number of shared birthdays is less than or equal to one" is "number of shared birthdays is greater than 1," which could also be worded "number of shared birthdays is at least 2," which satisfies this problem.\\
$\boxed{1 - ((\frac{366}{366}) (\frac{365}{366}) (\frac{364}{366}) ... (\frac{341}{366})) = .567684 = 56.768\%}$

\paragraph{4. }
\textbf{A coin is flipped 18 times. The result has the following pattern: one or more heads, one or more tails, one or more heads, one or more tails. Exactly one of the runs of heads has length at least eight. How many ways can this happen?}\\
We first enforced the rule of having at least one coin in each of the four groups. To do this we placed a ``springy'' gap after each of these coins.
\begin{verbatim}
H__T__H__T__
\end{verbatim}
The next rule is that there must be \textbf{exactly} one of the runs has length at least eight (See below details on the emphasis of the word ``exactly'').
In order to enforce this rule we group 7 H's into one token. Those 7, and the original 4, leave 7 more as the remainder. There are two groups which the 7 H's can be placed in so we will multiply the solution by two. Next we use the ``Stars and Bars'' formula to solve the counting of the 7 unknown coins.
\begin{verbatim}
H[HHHHHHH]___T___H___T___
           * | * | * | *
         or
H___T___H[HHHHHHH]___T___
  * | * | * | *
\end{verbatim}
The 7 unknown plus 3 bars gives us ${10 \choose 3}$. Calculating for both groups the 7 heads need to go in, the solution is:\\
$2{10 \choose 3} = 240$\\
But remember, there was only \textbf{one} run that could have length greater than or equal to 8. Out of the possible compositions it is possible for the 7 unknowns to all be heads meaning there would be \textbf{two} runs of 8. This means the actual solution must be:\\
$\boxed{2{10 \choose 3} - 1 = 239}$

\paragraph{5. }
\textbf{Consider the following pair of equations: $x_1 + x_2 + . . . + x_6 = 20$ and $x_1 + x_2 + x_3 = 7$. These equations must be simultaneously satisfied. How many non-negative, integer solutions are there?}\\
This can be easily simplified by substituting the 7 from the $2^{nd}$ equation for $x_1+x_2+x_3$ in the first formula. The result would be the following equations:\\
$x_1+x_2+x_3 = 7$ and $x_4+x_5+x_6 = 13$.
Using ``Stars and Bars'' on each of these the solution is:\\
 $\boxed{{9 \choose 2}{15 \choose 2} = 3780}$

\pagebreak

\paragraph{6. }
\textbf{Give a combinatorial argument for the following identity:}
\begin{center}
$\sum\limits_{k=0}^n {n \choose k}^2 = {2n \choose n}$
\end{center}

Two generals are taxed with building an invasion force. General 1 is in charge of the navy of which there are $n$ sailors. General 2 is in charge of the marines of which there are ${m = n}$ marines. 

General 1 wants to use the left side of the equation, ${\sum\limits_{k=0}^n {n \choose k}^2}$. In other words, general 1 would like to find the number of possible invasion forces where the chosen marines and sailors are equal. We can find this number by summing the counts of possible invasion forces based off of how many people are chosen from each group, ${ ({n \choose 0} {m \choose 0}) + ({n \choose 1} {m \choose 1}) + ... + ({n \choose n-1} {m \choose m-1}) + ({n \choose n} {m \choose m})}$.

General 2 wants to use the right side of the equation, ${2n \choose n}$. By putting all of the sailors and all of the marines into one group how many ways can you build an invasion force of $n$ people?

After much discussion general 2 proposes that they look at her solution in a new light. Out of all of the possible n size invasion forces the right side of the equation counts there are ${ {n \choose 0} {m \choose m}}$ ways in which the invasion force could be chosen with 0 sailors, ${ {n \choose 1} {m \choose m-1}}$ ways in which the force could be chosen with 1 sailor, and so on. Also, for every way to choose a group of sailors there is an equivalent way to choose those not added to the invasion force, i.e. ${ {n \choose 0} = {n \choose n}}$. So, if general 2 counts all of the number of ways to create the invasion force based off of how many sailors are in the force we get ${ ({n \choose 0} {m \choose m}) + ({n \choose 1} {m \choose m-1}) + ... + ({n \choose n-1} {m \choose 1}) + ({n \choose n} {m \choose 0})}$. We know that ${m = n}$ and each way to choose a number of soldiers has an equivalent way not to choose them which means that: 

${ ({n \choose 0} {m \choose m}) + ({n \choose 1} {m \choose m-1}) + ... + ({n \choose n-1} {m \choose 1}) + ({n \choose n} {m \choose 0})}$ 

${= ({n \choose 0} {m \choose 0}) + ({n \choose 1} {m \choose 1}) + ... + ({n \choose n-1} {m \choose m-1}) + ({n \choose n} {m \choose m})}$. 

So, both of the generals were searching for the same number.

\end{document}