% CS410 HW4

\documentclass{article}
\usepackage{anysize}
\usepackage{amsmath}
\usepackage{graphicx}

\marginsize{1.5cm}{1.5cm}{1.5cm}{1.5cm}

\title{CS410 HW4}
\author{Russell Miller}
\date{\today}

\begin{document}

\maketitle

% 2.14, 3.6, 2.21, 3.3, 3.21                                                    |

\paragraph{2.14 The geometric distrubution arises as the distribution of the 
number of times we flip a coin until it comes up heads. Consider now the 
distribution of the number of flips $X$ until the $k$th head appears, where 
each coin flip comes up heads independently with probability $p$. Prove that 
this distribution is given by}
\begin{eqnarray*}
\mbox{Pr}(X=n)={n-1 \choose k-1}p^k(1-p)^{n-k}\\
\mbox{for } n \geq k.
\end{eqnarray*}

We would like to know when the $k$th head is flipped. Let $X_1$ be a geometric 
random variable that is the distribution of the number of times we flip the coin
until the first heads, and so on for all values of $i=\{1,2,...,k\}$.
\begin{eqnarray*}
X = \sum\limits_{i=1}^k X_i
\end{eqnarray*}

For each of these geometric random variables $X_i$, the distribution for the 
$j$th head is
\begin{eqnarray*}
\mbox{Pr}(X_i=j) = (1-p)^{j-1}p 
\end{eqnarray*}

So the distribution of $X$ would be
\begin{eqnarray*}
\mbox{Pr}(X=n)=\sum\limits_{k=1}^n (1-p)^{k-1}p
\end{eqnarray*}

\Large SO LOST

\end{document}
