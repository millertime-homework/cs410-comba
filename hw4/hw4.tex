% CS410 HW4

\documentclass{article}
\usepackage{anysize}
\usepackage{amsmath}
\usepackage{amssymb}
\usepackage{graphicx}

\marginsize{1.5cm}{1.5cm}{1.5cm}{1.5cm}

\title{CS410 HW4}
\author{Russell Miller}
\date{\today}

\begin{document}

\maketitle

% 2.14, 3.6, 2.21, 3.3, 3.21                                                    |

\paragraph{2.14 The geometric distribution arises as the distribution of the 
number of times we flip a coin until it comes up heads. Consider now the 
distribution of the number of flips $X$ until the $k$th head appears, where 
each coin flip comes up heads independently with probability $p$. Prove that 
this distribution is given by}
\begin{eqnarray*}
\mbox{Pr}(X=n)={n-1 \choose k-1}p^k(1-p)^{n-k}\\
\mbox{for } n \geq k.
\end{eqnarray*}

The distribution of a binomial random variable is
\begin{eqnarray*}
Pr(X=k)={n\choose k}p^k(1-p)^{n-k}
\end{eqnarray*}
The difference between this and the aforementioned distribution is that this 
gives the probability that there are $k$ heads. What we want to find is the 
probability that after finding $k$ heads, we have only flipped $n$ coins.\\
\\
Assume that on the $n$th flip, we achieve our $k$th heads. Now we know that on 
the previous $n-1$ flips, we had exactly $k-1$ heads. So the probability that 
we tossed $k-1$ heads in $n-1$ flips is the same as the probability that the 
$k$th heads was achieved on the $n$th flip. However, $p$ and $p-1$ are not 
raised to the $k-1$, and that's because there are in fact $k$ total heads. 
$\blacksquare$

\paragraph{3.6 For a coin that comes up heads independently with probability
$p$ on each flip, what is the variance in the number of flips until the $k$th
head appears?\\}
The distribution of the number of flips of a coin until the $k$th heads could
be viewed as a sum of the distributions of geometric random variables 
representing the previous $k-1$ heads. Let $X$ be the random variable for the
number of coin flips until the $k$th heads, and each $X_i$ be a geometric 
random variable for the number of flips to get the $i$th heads. For example,
$X_1$ is the number of coin flips until the first heads. 
\begin{eqnarray*}
X = \sum_{i=1}^k X_i
\end{eqnarray*}
We know that if each $X_i$ is mutually independent (which we're told each coin
flip is)
\begin{eqnarray*}
Var[\sum_{i=1}^k X_i] = \sum_{i=1}^k Var[X_i]
\end{eqnarray*}
We need the variance of $X_i$, which we derived in class.
\begin{eqnarray*}
Var[X_i] = \frac{1-p}{p^2}
\end{eqnarray*}
Plugging this back into our distribution of X to find its variance
\begin{eqnarray*}
Var[X] & = & \sum_{i=1}^k \frac{1-p}{p^2}
\end{eqnarray*}
\begin{center}
$\boxed{=\frac{k(1-p)}{p^2}}$
\end{center}

\paragraph{2.21 Let $a_1, a_2, ..., a_n$ be a random permutation of 
$\{1, 2, ..., n\}$, equally likely to be any of the $n!$ possible permutations. 
When sorting the list $a_1, a_2, ..., a_n$, the element $a_i$ must move a 
distance of $|a-i|$ places from its current position to reach its 
position in the sorted order. Find}
\begin{eqnarray*}
E\left[\sum_{i=1}^n |a_i-1|\right],
\end{eqnarray*}
\textbf{the expected total distance that elements will have to be moved.\\\\}
Let $X_i$ be a random variable for the distance that element $i$ moves, 
$|a_i-i|$. Let $a_i$ be a random variable for the $i$th element's 
position in a random permutation.
\begin{eqnarray*}
E[a_i] = \sum_{i=1}^n i\mbox{ Pr}(a_i=i)
\end{eqnarray*}
Because any of the $n$ positions is equally likely,
\begin{eqnarray*}
\mbox{Pr}(a_i=i) & = & \frac{1}{n}\\
E[a_i] & = & \sum_{i=1}^n \frac{i}{n}\\
	& = & \frac{1}{n}(1+2+...+n)\\
	& = & \frac{1}{n}\frac{n(n+1)}{2}\\
	& = & \frac{n+1}{2}
\end{eqnarray*}
Now we can find $E[X_i]$ by the linearity of expectation.
\begin{eqnarray*}
E[X_i] & = & E[|a_i-i|]\\
	& = & |E[a_i]-E[i]|\\
	& = & \left|\frac{n+1}{2}-i\right|
\end{eqnarray*}
Now we are ready to find $E[X]$, and we'll use linearity of expectation again.
\begin{eqnarray*}
E[X] & = & E\left[\sum_{i=1}^n X_i\right]\\
	& = & \sum_{i=1}^n E[X_i]\\
	& = & \sum_{i=1}^n \left|\frac{n+1}{2}-i\right|\\
	& = & \left|\sum_{i=1}^n \frac{n+1}{2}-\sum_{i=1}^n i\right|\\
	& = & \left|\frac{n+1}{2}-\frac{n(n+1)}{2}\right|
\end{eqnarray*}
\begin{center}
$\boxed{\left|\frac{1-n^2}{2}\right|}$
\end{center}

\end{document}
