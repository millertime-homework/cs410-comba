% CS410 Midterm
% Take-Home portion

\documentclass{article}
\usepackage{anysize}
\usepackage{amsmath}
\usepackage{graphicx}

\marginsize{1.5cm}{1.5cm}{1.5cm}{1.5cm}

\title{CS410 Take-home Midterm}
\author{Russell Miller}
\date{\today}

\begin{document}

\maketitle

\begin{verbatim}
                                             __J"L__
                                         ,-"`--...--'"-.
                                        /  /\       /\  \
                                       J  /__\  _  /__\  L
                                       |       / \       |
                                       J    _  """  _    F
                                        \   \\/\_/\//   /
                                         "-._\/\_/\/_,-"
                                             """""""
\end{verbatim}

\paragraph{Textbook 2.22\\}
\textbf{Let $a_1, a_2, ..., a_n$ be a list of $n$ distinct numbers. We say that
$a_i$ and $a_j$ are $inverted$ if $i<j$ but $a_i>a_j$. The $Bubblesort$ sorting
algorithm swaps pairwise adjacent inverted numbers in the list until there are 
no more inversions, so the list is in sorted order. Suppose that the input to 
Bubblesort is a random permutation, equally likely to be any of the $n!$ 
permutations of $n$ distinct numbers. Determine the expected number of 
inversions that need to be corrected by Bubblesort.\\\\}
First we define the random variable $X$ to represent the number of inversions 
required for the list to be sorted. $X$ is made up of a set of indicator random
variables $X_{ij}$ where each $X_{ij} = 1$ when $a_i$ and $a_j$ are inverted. 
That is, $i<j$ and $a_i>a_j$. Since we know that
\begin{eqnarray*}
X & = & \sum\limits_{i<j} X_{ij}
\end{eqnarray*}

by the Linearity of Expectation, we know that
\begin{eqnarray*}
E[X] & = & \sum\limits_{i<j} E[X_{ij}]
\end{eqnarray*}

and since $j$ is defined as being adjacent to $i$
\begin{eqnarray*}
     & = & \sum\limits_{i=1}^n \sum\limits_{j=i+1}^n Pr(X_{ij}=1)
\end{eqnarray*}

The probability of a pair of numbers being inverted is simply $\frac{1}{2}$.
$a_i$ and $a_j$ are two random digits, and it's equally likely that either 
could be greater than the other.
\begin{eqnarray*}
Pr(X_{ij}) & = & \frac{1}{2}
\end{eqnarray*}

Plugging this back in
\begin{eqnarray*}
E[X] & = & \sum\limits_{i=1}^n \sum\limits_{j=i+1}^n \frac{1}{2}\\
     & = & \left(\frac{n}{2}\right)\left(\frac{n-1}{2}\right)
\end{eqnarray*}

\begin{center}
$\boxed{E[X] = \frac{n(n-1)}{4}}$
\end{center}

\end{document}
